\documentclass[a4wide, 11pt]{article}
\usepackage{a4, fullpage}
\setlength{\parskip}{0.3cm}
\setlength{\parindent}{0cm}

% This is the preamble section where you can include extra packages etc.

\begin{document}

\title{MAlice Language Specification}

\author{Magdalena Gocek \and Varun Verma \and Harshwardhan Ostwal}

% \date{\today}         % inserts today's date

\maketitle            % generates the title from the data above

\section{Introduction}
\label{sec:intro}
This is \emph{specification for MAlice}. This would define the language
constructs for the language MAlice. This specification consists of
the following things:
\begin{itemize}
	\item Language Syntax Definition
	\item Language Semantics Definition
\end{itemize}

%\textbf{quite easy to get started}.
%Try playing around with this file and see. 
%Don't worry that this page looks very spaced out,
%\LaTeX\ arranges the page for you (much less work to do than in Word!)
%so if there was more content it would close up the gaps.

%To start a new paragraph just leave a blank line.
%If you do not like numbered sections, use the \texttt{section*\{...\}}
%environment instead.

\section{BNF Grammar}
\label{sec:bnf}
	\begin{tabbing}
	\hspace*{9mm}\=\hspace*{25mm}\=\kill
		\> Identifier $\rightarrow$ [a-zA-Z][a-z]* \\
		\> Letter $\rightarrow$ [a-zA-Z] \\
		\> Number $\rightarrow$ [0-9]+ \\
		\> Data $\rightarrow$ Number $\mid$ Identifier \\
		\> Function $\rightarrow$ Drank $\mid$ Ate \\
		\> Expr	$\rightarrow$ Number $\mid$ Identifier $\mid$ MONOOP Expr 
					$\mid$ Data BINOP Data \{ BINOP Data \}\\
		\> BINOP $\rightarrow$ '+' $\mid$ '$\%$' $\mid$ '/' $\mid$ '\^{ }' 
							$\mid$ '$\&$' $\mid$ '$\mid$' $\mid$ '$\ast$' \\
		\> MONOOP $\rightarrow$ '$\mathtt{\sim}$'\\
		\\
		\> DataType $\rightarrow$ \emph{'number'} $\mid$ \emph{'letter'} \\
		\\
		\> Statement $\rightarrow$ Identifier \emph{'was a'} DataType \{ \emph{'too'} \}\\
		\> \> $\mid$ Identifier \emph{'became'} \{Letter $\mid$ Expr\} \\
		\> \>	$\mid$ Expr \emph{'said' 'Alice'} \\
		\> \> $\mid$ Identifier Function \\
	\end{tabbing}

\begin{verbatim}
\end{verbatim}

\section{Semantics}
\label{sec:semantics}
	\emph{MAlice} domain consists of following things which are described futher in the
		referenced sub sections:
	\begin{itemize}
		\item Data Types and Variables (See Section~\ref{sec:dtypes})
		\item Mathematical Operations (See Section~\ref{sec:mathOper})
		\item Predefined Functions (See Section~\ref{sec:preFunc})
		\item Different Types of Statements (See Section~\ref{sec:statements})
		\item The program Structure (See Section~\ref{sec:progStr}) 
	\end{itemize}
	The structure of the program written in \emph{MAlice} can be described using the above items. A simple
	\emph{MAlice} program makes use of the variables and functions which are building blocks
	for most of the widely used language. Each of the above component named about is described further
	in their respective sections. The limitations to each of the component is also defined.

	\subsection{Data Types and Variables}
	\label{sec:dtypes}
		\emph{MAlice} only consists of two basic data types which are:
		\begin{itemize}
			\item Letter (See~\ref{sec:letter})
			\item Number (See~\ref{sec:num})
		\end{itemize}
		These are the only two data types which can be supported by the language \emph{MAlice}. MAlice
		supports variables(See~\ref{sec:var}) as well.
		\subsubsection{Number}
			\label{sec:num}
			MAlice uses a 9-bit signed integer representation to represent the data type number. The data of type
			number in MAlice is stored in \emph{Two's Complement} representation. This uses the most signifincant
			bit is defined as a complement. 
			This contstrains the data type number to have a specified range of \emph{-256 to 255}. This
			contrain also introduces some errors which would need to be dealt with during compile time as
			well as run time. The error related to type number which could be encountered are as follows:
			\begin{itemize}
				\item {\bf OutOfBoundError}: This error can be encoutered at run time. 
							This error is caused when during
							runtime a value for the data type number is outwith the range (\emph{-256 to 255}).
				\item {\bf InvalidAssignment}: This error would be encountered if a 
							variable of type number has been assigned a value of number which is 
							outwith the range of the data type number. 
			\end{itemize}
	
	\subsubsection{Letter}
		\label{sec:letter}
			The letter data type can contain ASCII characters only within 'a' to 'z' and 'A' and 'Z'.
			This data type would use the respective ASCII representation to store the character.

	\subsubsection{Variables}
		\label{sec:var}
			Varibles in MAlice have an String identifier like in most other programming languages.
			The identifier for a variable can only contains letters (i.e. 'a' to 'z' and 'A' to 'Z'.
			The first character can be either upper case or lower case but the following character has to be
			lower case. For example, "TExt" and "MyVar" are not valid identifiers whereas "Variable" and
			"variable" are valid. Variable can be of type 
			\emph{number or letter} in MAlice. The errors related to Variables which could be encountered are:
			\begin{itemize}
				\item {\bf InvalidAssignment}: A variable of type 'number' can not be assigned to a variable which
					has a type 'letter' or vice versa.
				\item {\bf ***NAMETHISERROR***}: An identifier name which has already been initialised can not
					be used again. All identifier names have to be unique.
			\end{itemize}
	
	\subsection{Mathematical Operations}
	\label{sec:mathOper}
		MAlice supports various different mathematical operations. The functionality of all the mathematical
		operations supported by MAlice is described below:
		\begin{itemize}
			\item {\bf '+'}: This is mathematical add operator and returns the sum of the given two arguments.
			\item {\bf '$\%$'}: This is the modulus operator and returns the modulus for the given two arguments.
			\item {\bf '/'}: This is the 
		\end{itemize}
 
	\subsection{Predefined Functions}
	\label{sec:preFunc}
	This is predefined sunctions

	\subsection{Statements}
	\label{sec:statements}
	This is statement

	\subsection{The Program Structure}
	\label{sec:progStr}
	This is the structure of the program.

	\section{Assumption}
	\label{sec:assumption}
	\begin{itemize}
		\item No variables are allowed with a digit in their name.
		\item	Variable can start with either a lower case or uppper case character but any character
					in the name of the variable can be upper case.
		\item The number data type in \emph{MAlice} is a \emph{9-bit} signed integer.
					This defined the limit of the number data type in Malice to be -256 to 255.
					Any other assignment outside these bouds would cause a \emph{RunTime Error}.
	\end{itemize}

Finally for this document, if you want to include a reference
then you put it into a \texttt{thebibliography\{...\}}
environment (see below in source file) and then 
cite it like this \cite{lamport94}
(you will need to run \texttt{latex} twice to get it to process the citation),
or you can use BibTex but that is probably overkill for now.

\begin{thebibliography}{9}

\bibitem{lamport94}
  Leslie Lamport,
  \emph{\LaTeX: A Document Preparation System}.
  Addison Wesley, Massachusetts,
  2nd Edition,
  1994.

\end{thebibliography}


\end{document}
