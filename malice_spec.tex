\documentclass[a4wide, 11pt]{article}
\usepackage{a4, fullpage}
\usepackage{listings}
\setlength{\parskip}{0.3cm}
\setlength{\parindent}{0cm}

% This is the preamble section where you can include extra packages etc.

\begin{document}

\title{MAlice Language Specification}

\author{Magdalena Gocek \and Varun Verma \and Harshwardhan Ostwal}

\maketitle            % generates the title from the data above

\section{Introduction}
\label{sec:intro}
This specification defines the language constructs of MAlice, focusing on  
the syntax and the semantics of the language. The syntax will be defined
using the Extended Backus-Naur Form (EBNF) grammar and Regular Expressions form.

%\textbf{quite easy to get started}.
%Try playing around with this file and see. 
%Don't worry that this page looks very spaced out,
%\LaTeX\ arranges the page for you (much less work to do than in Word!)
%so if there was more content it would close up the gaps.

%To start a new paragraph just leave a blank line.
%If you do not like numbered sections, use the \texttt{section*\{...\}}
%environment instead.

\section{Syntax Definition}
\label{sec:bnf}
	\begin{tabbing}
	\hspace*{8mm}\=\hspace*{25mm}\=\hspace*{15mm}\=\hspace*{2mm}\=\kill
		\> Number $\rightarrow$ [0-9]+ \\
		\> Letter $\rightarrow$ '[a-zA-Z]' \\
		\> Identifier $\rightarrow$ [a-zA-Z][a-z]$\ast$ \\
		\> FunctionName $\rightarrow$ Identifier \\
		\> Data $\rightarrow$ Number $\mid$ Identifier \\
		\> Expr	$\rightarrow$ MONOOP Expr $\mid$ Data ( $\epsilon$ $\mid$ BINOP Expr )\\
		\> BINOP $\rightarrow$ '+' $\mid$ '$\%$' $\mid$ '/' $\mid$ '\^{ }' 
							$\mid$ '$\&$' $\mid$ '$\mid$' $\mid$ '$\ast$' \\
		\> MONOOP $\rightarrow$ '$\mathtt{\sim}$'\\

		\> DataType $\rightarrow$ \emph{'number'} $\mid$ \emph{'letter'} \\
		
		\> Statement $\rightarrow$ Identifier ( \emph{'was a'} DataType [ \emph{'too'} ]\\
		\> \> \> \> $\mid$  \emph{'became'} ( Letter $\mid$ Expr ) \\
		\> \> \> \> $\mid$  (FunctionName $\mid$ \emph{'ate'} $\mid$ \emph{'drank'} ) \\
		\> \> \> )  \\
		\> \>	$\mid$ Expr \emph{'said' 'Alice'} \\
	
				
		\> StatementConjunctions $\rightarrow$ \emph{'.'} $\mid$ \emph{'and'} $\mid$ \emph{'but'} $\mid$ \emph{','} $\mid$ \emph{'then'} \\
		\> StatementList $\rightarrow$ $\epsilon$ $\mid$ Statement \{StatementConjunctions Statement\} '.' \\

		\> Function $\rightarrow$ \emph{'The looking-glass'} FunctionName  \emph{'()' 'opened' } StatementList '\emph{closed}' \\
		\> Program $\rightarrow$ \{Function\}
	\end{tabbing}
	

\begin{verbatim}
\end{verbatim}

\section{Semantics}
\label{sec:semantics}
	A MAlice program makes use of variables and functions which are building blocks for most of the widely used programming
	languages. It consists of the following components which have been described further in respective sub-sections
	together with their limitations:
	\begin{itemize}
		\item Data Types
		\item Variables 
		\item Mathematical Operations 
		\item Functions
		\item Statements
	\end{itemize}
	

	\subsection{Data Types}
	\label{sec:dtypes}		
		MAlice supports two basic data types, \emph{number} and \emph{letter}. The letter 
		data type is represented by a single ASCII character enclosed in single quotes (' ') ranging from 'a' to 'z' and 
		'A' to 'Z'. The number data type is represented by a 16-bit signed integer. MAlice uses the 
		Two's Copmlement representation for the number data type which defines
		the most significant bit to be the complement. This constraints the data type number 
		to have a specified range of \emph{-65536 to 65535} and introduces the following errors: 

			\begin{itemize}
				\item {\bf InvalidAssignment}: This error would be encountered if the value assigned to a variable
							is out of the range of data type number.
				\item {\bf IntegerOverflow}: This error would be encountered if the accumulating result or the final result
							of an expression is out of the range of the integer. 
			\end{itemize}
	
	\subsection{Variables}
		\label{sec:var}
			Variables in MAlice have a String identifier which consists of letters only.
			The identifier can start with upper or lower case letter and all the 
			following letters must be lower case. For example, "TExt" and "MyVar" are not 
			valid identifiers whereas "Variable" and "variable" are valid. Variables 
			can hold a value of a type \emph{number} or \emph{letter}. The possible errors encountered
			when using variables include:
			\begin{itemize}
				\item {\bf InvalidAssignment}: A variable of type \emph{number} cannot be 
					assigned to a variable of a type \emph{letter} and vice versa.
				\item {\bf InvalidInitialisation}: An identifier name for a variable which 
					has already been initialised cannot	be used again. All identifier names 
					must be unique.
			\end{itemize}
	
	\subsection{Mathematical Operations}
	\label{sec:mathOper}
		MAlice supports various mathematical operations, the functionality of which is 
		described below with their precedence shown in brackets:
		\begin{itemize}
			\item {\bf '$\mathtt{\sim}$'(2)}: Bitwise NOT. The return value is obtained by carrying
				out logical NOT operation on individual bits of the given argument.
			\item {\bf '$\ast$'(3)}: Mathematical multiply - returns the product of the
				given arguments.
			\item {\bf '/'(3)}: Division operator. A runtime error would be encountered 
				when trying to	divide by 0.
			\item {\bf '$\%$'(3)}: Modulus operator - returns the reminder after division of 
				the given arguments.
			\item {\bf '+'(4)}: Mathematical add - returns the sum of the given arguments.
			\item {\bf '$\&$'(8)}: Bitwise AND - the return value is obtained by carrying out 
				logical AND operation on individual bits of the given arguments.
			\item {\bf '\^{ }'(9)}: Bitwise XOR - the return value is obtained by carrying out 
				logical XOR operation on individual bits of the given arguments.
			\item {\bf '$\mid$'(10)}: Bitwise OR - the return value is obtained by carrying out 
				logical OR operation on the individual bits of the given arguments.
		\end{itemize}
 
	\subsection{Functions}
	\label{sec:func}
		MAlice has two built-in functions, \emph{'ate'} and \emph{'drank'}. These
		functions respectively increment and decrement the value of the given argument by 1 and 
		can only be applied to variables of a type number.
		MAlice also supports user-defined functions which can be used in any other functions
		within the program. Each user-defined function must start with the keywords
		\emph{'The looking-glass'} followed by the name of the function with empty brackets. The user can 
	 	specify the parameters in the brackets.	The function body is enclosed within the keywords \emph{'opened'}
		and \emph{'closed'}. User defined functions in MAlice can accept either 1 or no 
		arguments which are passed by value. The program starts with the function \emph{'hatta'}. An example is shown below:
		\begin{verbatim}
		The looking-glass Functionname(parameter)
		opened
		  {body}
		closed
		\end{verbatim}
	\newpage

	\subsection{Statements}
	\label{sec:statements}
		A MAlice program consists of a sequence of various statements. This can be of one of the following forms:
		\begin{itemize}
			\item {\bf Initialisation Statement}: The initialisation in MAlice is done by choosing an identifier
		 		for the variable and then defining its data type by using the keyword \emph{'was a'}, for example:
				\begin{verbatim}
					Seven was a number. A was a letter.
				\end{verbatim}
				Here 'Seven' and 'A' are the identifieres for two variables of type 
				\emph{letter} and \emph{number} respectively. The identifiers will be 
				used to access these variables in the function they are defined in. 
			\item {\bf Assignment Statement}: The assignment can be done by stating the 
				variable identifier and using the keyword \emph{'became'} followed by a letter, variable, number or 
				mathematical expression, for example:  
				\begin{verbatim}
					Seven became 2. A became 'e'. Seven became Seven*4.
				\end{verbatim}
			\item {\bf Print Statement}: The print statement is used to output to the standard output. 
				The print statement is invoked by stating the variable identifier, letter, number
				or a mathematical expresion followed by the key phrase \emph{'said Alice'}. For example, the following statement
				would print 24 to standard output.
				\begin{verbatim}
					6*4 said Alice.
				\end{verbatim}
			\item {\bf Function Statement}: The function statement is only used when there is a need to apply 
				one of the MAlice's predefined functions or the functions defined in the program. To do this, we state a number or a variable's identifier 
				followed by the name of the function, for example:
				\begin{verbatim}
					Variable drank. Seven ate. Variable arbitaryFunction.
				\end{verbatim}
		\end{itemize}
		The statements in a program are separated by the following keywords and symbols: 
		\{ \emph{'.'} , \emph{','} , \emph{'and'} , \emph{'then'} , \emph{'but'}\}.
		Each statement is executed in sequential order. Any statement which does not comply with
		the rules of BNF would cause a \emph{SyntaxError}.

\end{document}
