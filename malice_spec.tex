\documentclass[a4wide, 11pt]{article}
\usepackage{a4, fullpage}
\setlength{\parskip}{0.3cm}
\setlength{\parindent}{0cm}

% This is the preamble section where you can include extra packages etc.

\begin{document}

\title{MAlice Language Specification}

\author{Magdalena Gocek \and Varun Verma \and Harshwardhan Ostwal}

% \date{\today}         % inserts today's date

\maketitle            % generates the title from the data above

\section{Introduction}
\label{sec:intro}
This is \emph{specification for MAlice}. This would define the language
constructs for the language MAlice. This specification consists of
the following things:
\begin{itemize}
	\item Language Syntax Definition
	\item Language Semantics Definition
\end{itemize}

%\textbf{quite easy to get started}.
%Try playing around with this file and see. 
%Don't worry that this page looks very spaced out,
%\LaTeX\ arranges the page for you (much less work to do than in Word!)
%so if there was more content it would close up the gaps.

%To start a new paragraph just leave a blank line.
%If you do not like numbered sections, use the \texttt{section*\{...\}}
%environment instead.

\section{BNF Grammar}
\label{sec:bnf}
	\begin{tabbing}
	\hspace*{9mm}\=\hspace*{25mm}\=\kill
		\> Identifier $\rightarrow$ [a-zA-Z][a-z]* \\
		\> Letter $\rightarrow$ [a-zA-Z] \\
		\> Number $\rightarrow$ [0-9]+ \\
		\> Data $\rightarrow$ Number $\mid$ Identifier \\
		\> Function $\rightarrow$ Drank $\mid$ Ate \\
		\> Expr	$\rightarrow$ Number $\mid$ Identifier $\mid$ MONOOP Expr 
					$\mid$ Data BINOP Data \{ BINOP Data \}\\
		\> BINOP $\rightarrow$ '+' $\mid$ '$\%$' $\mid$ '/' $\mid$ '\^{ }' 
							$\mid$ '$\&$' $\mid$ '$\mid$' $\mid$ '$\ast$' \\
		\> MONOOP $\rightarrow$ '$\mathtt{\sim}$'\\
		\> DataType $\rightarrow$ \emph{'number'} $\mid$ \emph{'letter'} \\
		\\
		\> Statement $\rightarrow$ Identifier \emph{'was a'} DataType \{ \emph{'too'} \}\\
		\> \> $\mid$ Identifier \emph{'became'} \{Letter $\mid$ Expr\} \\
		\> \>	$\mid$ Expr \emph{'said' 'Alice'} \\
		\> \> $\mid$ Identifier Function \\
		\> StatementConjunctions $\rightarrow$ '.' $\mid$ 'and' $\mid$ 'but' $\mid$ ',' $\mid$ 'then' \\
		\> StatementList $\rightarrow$ EPSILON** \\ 
		\> \> $\mid$ Statement StatementConjuctions Statement \{StatementConjunctions Statement\} \\
	 	\\
		\> Program $\rightarrow$ \emph{'The' 'looking-glass' 'hatta' 'opened'} StatementList '\emph{closed}'
	\end{tabbing}

\begin{verbatim}
\end{verbatim}

\section{Semantics}
\label{sec:semantics}
	\emph{MAlice} domain consists of following things which are described futher in the
		referenced sub sections:
	\begin{itemize}
		\item Data Types and Variables (See Section~\ref{sec:dtypes})
		\item Mathematical Operations (See Section~\ref{sec:mathOper})
		\item Predefined Functions (See Section~\ref{sec:preFunc})
		\item Different Types of Statements (See Section~\ref{sec:statements})
		\item The program Structure (See Section~\ref{sec:progStr}) 
	\end{itemize}
	The structure of the program written in \emph{MAlice} can be described using the above items. A simple
	\emph{MAlice} program makes use of the variables and functions which are building blocks
	for most of the widely used language. Each of the above component named about is described further
	in their respective sections. The limitations to each of the component is also defined.

	\subsection{Data Types and Variables}
	\label{sec:dtypes}
		\emph{MAlice} only consists of two basic data types which are:
		\begin{itemize}
			\item Letter (See~\ref{sec:letter})
			\item Number (See~\ref{sec:num})
		\end{itemize}
		These are the only two data types which can be supported by the language \emph{MAlice}. MAlice
		supports variables(See~\ref{sec:var}) as well.
		\subsubsection{Number}
			\label{sec:num}
			MAlice uses a 9-bit signed integer representation to represent the data type number. The data of type
			number in MAlice is stored in \emph{Two's Complement} representation. This uses the most signifincant
			bit is defined as a complement. 
			This contstrains the data type number to have a specified range of \emph{-256 to 255}. This
			contrain also introduces some errors which would need to be dealt with during compile time as
			well as run time. The error related to type number which could be encountered are as follows:
			\begin{itemize}
				\item {\bf OutOfBoundError}: This error can be encoutered at run time. 
							This error is caused when during
							runtime a value for the data type number is outwith the range (\emph{-256 to 255}).
				\item {\bf InvalidAssignment}: This error would be encountered if a 
							variable of type number has been assigned a value of number which is 
							outwith the range of the data type number. 
			\end{itemize}
	
	\subsubsection{Letter}
		\label{sec:letter}
			The letter data type can contain ASCII characters only within 'a' to 'z' and 'A' and 'Z'.
			This data type would use the respective ASCII representation to store the character.

	\subsubsection{Variables}
		\label{sec:var}
			Varibles in MAlice have an String identifier like in most other programming languages.
			The identifier for a variable can only contains letters (i.e. 'a' to 'z' and 'A' to 'Z'.
			The first character can be either upper case or lower case but the following character has to be
			lower case. For example, "TExt" and "MyVar" are not valid identifiers whereas "Variable" and
			"variable" are valid. Variable can be of type 
			\emph{number or letter} in MAlice. The errors related to Variables which could be encountered are:
			\begin{itemize}
				\item {\bf InvalidAssignment}: A variable of type 'number' can not be assigned to a variable which
					has a type 'letter' or vice versa.
				\item {\bf ***NAMETHISERROR***}: An identifier name which has already been initialised can not
					be used again. All identifier names have to be unique.
			\end{itemize}
	
	\subsection{Mathematical Operations}
	\label{sec:mathOper}
		MAlice supports various different mathematical operations. The functionality of all the mathematical
		operations supported by MAlice is described below:
		\begin{itemize}
			\item {\bf '+'}: This is mathematical add operator and returns the sum of the given two arguments.
			\item {\bf '$\%$'}: This is the modulus operator and returns the modulus for the given two arguments.
			\item {\bf '/'}: This is the division operator. A runtime error would be encountered while trying to
				divide by 0.
			\item {\bf '\^{ }'}: This bitwise XOR operator. The return value is bitwise XOR operation on each
				individual bits of the given two arguments.
			\item {\bf '$\&$'}: This is bitwise AND operator. This return value is obtained by carrying out 
				logical AND operation on individual bits of the given two arguments.
			\item {\bf '$\mid$'}: This is bitwise OR operation. This returns value is obtained by carrying out 
				logical OR operation on the individual bits of the given two arguments.
			\item {\bf '$\ast$'}: This is mathematical multiply operator. The return value is the product of the
				given two arguments.
			\item {\bf '$\mathtt{\sim}$'}: This is bitwise NOT operator. The return value is obtained by carrying
				out logical NOT operation on individual bits of the given argument.
		\end{itemize}
 
	\subsection{Predefined Functions}
	\label{sec:preFunc}
		MAlice only has two built-in functions. These functions are \emph{"drank"} and \emph{"ate"}. The function
		\emph{"drank"} decrements the value of the given argument by 1 and \emph{"ate"} increments the value of the
		given argument by 1. These functions can only be applied to variables of type number.

	\subsection{Statements}
	\label{sec:statements}
		MAlice supports the follwing type of programming statments:
		\begin{itemize}
			\item {\bf Initialisation Statement}: The initialisation in MAlice is done by using the identifier
		 		for the variable and then defining its data type as follows:
				\begin{verbatim}
					Identifier was a number. Variable was a letter.
				\end{verbatim}	
				The Identifer and Variable are the names which would be used to access these variables throughout
				the program and 'letter' and 'number' are the data types of each of the variables.
			\item {\bf Assignment Statement}: The assignment can be done either using other letters, variables, numbers,
				mathematical operations or by applying functions.
				\begin{verbatim}
					Identifier became 2. Variable became 'e'. Identifier became Identifier*4.
					Identifier drank.
				\end{verbatim}
				The keyword \emph{"became"} initialises the given variable with the specified value, for example,
				'e' in the second case and 2 for the first case.
			\item {\bf Print Statement}: The print statement is used to output to the standard output. The print
				statment is as follows which print 24 to standard output:
				\begin{verbatim}
					Variable said Alice. 6*4 said Alice.
				\end{verbatim}
			\item {\bf Function Statement}: There are only two functions which are in domain of MAlice which are
				\emph{"drank"} and \emph{"ate"}. These functions are applied on variables and the statement for these 
				functions can only be where 'drank' and then 'ate' is applied to Variable:
				\begin{verbatim}
					Variable drank. Variable ate.	
				\end{verbatim}
		\end{itemize}
		These statements can be joined using the one of the following keywords or symbols, \{ \emph{'.' , ',' , 'and' , 'then' , 'but'}\}.

	\subsection{The Program Structure}
	\label{sec:progStr}
		The program in MAlice starts with \emph{"The looking-glass hatta ()"} and next line the keyword \emph{"opened"}
		and then code body of program in MAlice and then the final keyword \emph{"closed"}.
\end{document}
