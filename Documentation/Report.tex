\documentclass[a4wide, 11pt]{article}
\usepackage{a4, fullpage}
\usepackage{listings}
\setlength{\parskip}{0.3cm}
\setlength{\parindent}{0cm}

% This is the preamble section where you can include extra packages etc.

\begin{document}

\title{MAlice Compiler Project}

\author{Magdalena Gocek \and Varun Verma \and Harshwardhan Ostwal}

\maketitle            % generates the title from the data above

\section{Introduction}
\label{sec:intro}
	This specification provides an overview for the project which involved building a compiler for language specification MAlice.
	
\section{Overview}
	\subsection{Analysis}
		The compiler which has been built allows the user to provide a MAlice(*.alice) file and generate equivalent code
		for the Malice file in llvm bytecode or x86 Assembly Instructions. It supports the functionality which was specified in
		the specification for the language malice. The tools which have been used in development for this compiler are LLVM and ANTLR.
		The scope of these tools have been described further in there respective sections.
		\subsubsection{ANTLR (ANother Tool for Language Recognition )}
			We used ANTLR to perform syntactic and semantic analysis. It allows the user to define the grammar in a 
			specific syntax and generates Lexer and Parser classes for this grammar in a specified output language.
			ANTLR uses Top-Down LR parsing. Due to this, backtracking option has to be enabled. This 
			compromises with the performance. This is the most straightforward way for parsing but at the same
			time it also slows down the performance. When the backtracking option is enabled and the parsing fails during
			the execution of a defined rule in the grammar, the parser has to track back to the previous node and check if there are
			any other viable routes which could be taken to parse the given input. Because of the way ANTLR parses the grammar, it was not the most suitable tool for this project.
			On the other hand ANTLR does provide some user-friendly graphical tools to verify the grammar, but they are not always accurate.
			
		\subsubsection{LLVM (Low Level Virtual Machine)}
			We used LLVM as the target language for the MAlice compiler. This is a widely used compiler's back-end tool which
			allows the user to output to various different architectures, such as ARM and x86. LLVM provides the user
			with three different optimisation levels. We opted for this tool because
			it is widely-used and has a very active development group. Which is why LLVM would be able to support new architectures as long as they have backwards compatibility. This means that our compiler will be able to operate on x86 and various other machines.
		
	\subsection{Functionality}			
		This compiler currently supports all the statements from the MAlice language specification. It provides
		the support for loops, conditional statements, functions, procedures and mathematical operations. Currently, there are some bugs
		occurring in array processing, nested functions and code blocks. The problem which has been encountered for 
		nested functions and code blocks are because it requires Lambda Lifting which requires a free variable analysis. This is currently
		not been supported by the compiler. Another performance issue which has been noticed is due to string comparison for the tokens in
		the nodes of the Abstract Syntax Tree (AST). The tree which ANTLR produces stores the nodes with a text field for the token.
		This affects significantly on the performance due to string comparison being highly resource expensive in terms of processing and memory.
		If the tree node were self-built then tokens, which would be an enumerated type would be used.
	
	\subsection{Optimisations}
		Malice compiler supports following optimisations:
		\begin{itemize}
			\item Evaluation of constant expression
			\item Elimination of unreachable code
		\end{itemize} 
		Optimisation for constant expressions evaluation is done by evaluating the values for each of the argument for a given operator in a function
		and then if they evaluate to an integer then the operator is applied and the evaluated result is returned. The optimisation
		for elimination of dead code is done by a similar approach to the evaluation of the constant expressions. Currently 
		the internal representation of the compiler treats an integer value 1 as true and 0 to be false. The type of expression has
		previously been checked in the semantics analysis part which ensure that if a given expression returns 1 then it is not 
		evaluated integer value 1.
	
		
\section{Design}
	The design uses a module structure to allow each individual component only carrying out one task which makes it easier to carry out
	component level testing. Most of the component which have been written have their own test suite which uses JUnit4 for the 
	environment to run the test suites. The design is very well structured, different packages for different parts of the 
	compilers are used. It can be easily seen what replacing any class with a new or different version of that component would require
	minimal modifications.
	\\
	\\
	Testing has been a major part of the development of the compiler since it has been able to direct in the correct direction
	for existence of a bug. It allowed the development part to produce more robust and reliable product.

\section{Extensions}
	Malice compiler has been implemented with three extensions which have been tested. Some of these are still not working for all cases 
	and would require some more of testing and development in order to reach a stable working version. These extensions vary vastly from
	modifying grammar to changing the generated abstract syntax tree using the current supported functionality but with easier syntax for 
	the user.
	\begin{itemize}
		\item Import Statements
		\item Higher Order Function Map
		\item Mathematical library
	\end{itemize}

	\subsection{Import Statement}
		Malice supports imports statements in the following format:
		\begin{verbatim}
		    Alice wants <function names> from <filename>.
		    Alice wants everything from "filename".
		    Alice wants function1, function2 from <filename>.
		\end{verbatim}
		The first statement describes the basic structure of the import statement. The import statements can only be added as an header for a 
		Malice file. Keyword "everything" can be used which imports all the global declarations and functions from the given file. 
		Currently to import any global declarations from a Malice file can only be done using the keyword "everything". Only functions can be
		imported individually and the user would have to make sure that any global statements which might be required are either added to the 
		file otherwise the program would not pass the semantics check. 
		\\
		\\
		Malice compiler checks for cyclic dependencies by using a depth first search implementation. The file which has been imported is stored
		in a Hash Set data structure and if any previous seen file is again encountered then an error message specifying the cyclic dependencies
		is printed. This extension allows a user to write modularised code as well as provides a better collaboration with the team, such as
		different people in the team can work on different modules in Malice. It also allows user to write their own standard library which they
		use quite often.

	\subsection{Higher Order Function Map}
		Malice supports Map function in the following format:
		\begin{verbatim}
		    map(functionName, input, output).
		\end{verbatim}
		Map(map) function in malice requires three arguments. The first argument is name of a single arity function. This function has
		to have parameter of the same type as the type of element of the 'input' array. This function when applied to a given value should
		return a value of type which has to be same as the type for the element of the output array. The function map in Malice is translated
		into a while loop which executes the calls for function call iteratively with help of a variable to keep track of the current index.
		
	\subsection{Mathematical Library}
		Malice compiler now supports a built in library for the mathematical functions which are very commonly used in a lot of widely
		used programming languages. The functions which are supported are all single arity functions and include "sin", "cos", "tan", "arctan"
		, "arcsin", "arccos", "cosh", "sinh", "tanh", "exp" and "sqrt". After building the library it was observed that we would require 
		another type to represent a floating point number in malice in order to use some of these functions. For example `arcsin` requires an
		argument between -1 and 1 and since in Malice we only have number type to represent integer, so we can only use one value which is
		0. Currently a float type has not yet been added but in later stages of the development adding another type to represent floating
		point number would allow the use of the functions.
		
		
\section{Summary}
	Git was used for version control of the project. It allowed smoother team work by providing each individual in the team to work
	on a different part of the project. One of the main issues which has been encountered has been while using ANTLR. Recently a new
	version of ANTLR has been released and it seemed to have quite a lot of bugs which caused us a lot of delays in generating the AST.
	The lack of analysis on each of individual tools which could have been used caused quite a lot of problems. ANTLR is very straight
	forward tool to use but it does has its limitations and its not the most suitable tool for this language.

\end{document}
